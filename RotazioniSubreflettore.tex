
% Default to the notebook output style

    


% Inherit from the specified cell style.




    
\documentclass{article}

    
    
    \usepackage{graphicx} % Used to insert images
    \usepackage{adjustbox} % Used to constrain images to a maximum size 
    \usepackage{color} % Allow colors to be defined
    \usepackage{enumerate} % Needed for markdown enumerations to work
    \usepackage{geometry} % Used to adjust the document margins
    \usepackage{amsmath} % Equations
    \usepackage{amssymb} % Equations
    \usepackage[mathletters]{ucs} % Extended unicode (utf-8) support
    \usepackage[utf8x]{inputenc} % Allow utf-8 characters in the tex document
    \usepackage{fancyvrb} % verbatim replacement that allows latex
    \usepackage{grffile} % extends the file name processing of package graphics 
                         % to support a larger range 
    % The hyperref package gives us a pdf with properly built
    % internal navigation ('pdf bookmarks' for the table of contents,
    % internal cross-reference links, web links for URLs, etc.)
    \usepackage{hyperref}
    \usepackage{longtable} % longtable support required by pandoc >1.10
    \usepackage{booktabs}  % table support for pandoc > 1.12.2
    

    
    
    \definecolor{orange}{cmyk}{0,0.4,0.8,0.2}
    \definecolor{darkorange}{rgb}{.71,0.21,0.01}
    \definecolor{darkgreen}{rgb}{.12,.54,.11}
    \definecolor{myteal}{rgb}{.26, .44, .56}
    \definecolor{gray}{gray}{0.45}
    \definecolor{lightgray}{gray}{.95}
    \definecolor{mediumgray}{gray}{.8}
    \definecolor{inputbackground}{rgb}{.95, .95, .85}
    \definecolor{outputbackground}{rgb}{.95, .95, .95}
    \definecolor{traceback}{rgb}{1, .95, .95}
    % ansi colors
    \definecolor{red}{rgb}{.6,0,0}
    \definecolor{green}{rgb}{0,.65,0}
    \definecolor{brown}{rgb}{0.6,0.6,0}
    \definecolor{blue}{rgb}{0,.145,.698}
    \definecolor{purple}{rgb}{.698,.145,.698}
    \definecolor{cyan}{rgb}{0,.698,.698}
    \definecolor{lightgray}{gray}{0.5}
    
    % bright ansi colors
    \definecolor{darkgray}{gray}{0.25}
    \definecolor{lightred}{rgb}{1.0,0.39,0.28}
    \definecolor{lightgreen}{rgb}{0.48,0.99,0.0}
    \definecolor{lightblue}{rgb}{0.53,0.81,0.92}
    \definecolor{lightpurple}{rgb}{0.87,0.63,0.87}
    \definecolor{lightcyan}{rgb}{0.5,1.0,0.83}
    
    % commands and environments needed by pandoc snippets
    % extracted from the output of `pandoc -s`
    \DefineVerbatimEnvironment{Highlighting}{Verbatim}{commandchars=\\\{\}}
    % Add ',fontsize=\small' for more characters per line
    \newenvironment{Shaded}{}{}
    \newcommand{\KeywordTok}[1]{\textcolor[rgb]{0.00,0.44,0.13}{\textbf{{#1}}}}
    \newcommand{\DataTypeTok}[1]{\textcolor[rgb]{0.56,0.13,0.00}{{#1}}}
    \newcommand{\DecValTok}[1]{\textcolor[rgb]{0.25,0.63,0.44}{{#1}}}
    \newcommand{\BaseNTok}[1]{\textcolor[rgb]{0.25,0.63,0.44}{{#1}}}
    \newcommand{\FloatTok}[1]{\textcolor[rgb]{0.25,0.63,0.44}{{#1}}}
    \newcommand{\CharTok}[1]{\textcolor[rgb]{0.25,0.44,0.63}{{#1}}}
    \newcommand{\StringTok}[1]{\textcolor[rgb]{0.25,0.44,0.63}{{#1}}}
    \newcommand{\CommentTok}[1]{\textcolor[rgb]{0.38,0.63,0.69}{\textit{{#1}}}}
    \newcommand{\OtherTok}[1]{\textcolor[rgb]{0.00,0.44,0.13}{{#1}}}
    \newcommand{\AlertTok}[1]{\textcolor[rgb]{1.00,0.00,0.00}{\textbf{{#1}}}}
    \newcommand{\FunctionTok}[1]{\textcolor[rgb]{0.02,0.16,0.49}{{#1}}}
    \newcommand{\RegionMarkerTok}[1]{{#1}}
    \newcommand{\ErrorTok}[1]{\textcolor[rgb]{1.00,0.00,0.00}{\textbf{{#1}}}}
    \newcommand{\NormalTok}[1]{{#1}}
    
    % Define a nice break command that doesn't care if a line doesn't already
    % exist.
    \def\br{\hspace*{\fill} \\* }
    % Math Jax compatability definitions
    \def\gt{>}
    \def\lt{<}
    % Document parameters
    \title{RotazioniSubreflettore}
    
    
    

    % Pygments definitions
    
\makeatletter
\def\PY@reset{\let\PY@it=\relax \let\PY@bf=\relax%
    \let\PY@ul=\relax \let\PY@tc=\relax%
    \let\PY@bc=\relax \let\PY@ff=\relax}
\def\PY@tok#1{\csname PY@tok@#1\endcsname}
\def\PY@toks#1+{\ifx\relax#1\empty\else%
    \PY@tok{#1}\expandafter\PY@toks\fi}
\def\PY@do#1{\PY@bc{\PY@tc{\PY@ul{%
    \PY@it{\PY@bf{\PY@ff{#1}}}}}}}
\def\PY#1#2{\PY@reset\PY@toks#1+\relax+\PY@do{#2}}

\expandafter\def\csname PY@tok@gd\endcsname{\def\PY@tc##1{\textcolor[rgb]{0.63,0.00,0.00}{##1}}}
\expandafter\def\csname PY@tok@gu\endcsname{\let\PY@bf=\textbf\def\PY@tc##1{\textcolor[rgb]{0.50,0.00,0.50}{##1}}}
\expandafter\def\csname PY@tok@gt\endcsname{\def\PY@tc##1{\textcolor[rgb]{0.00,0.27,0.87}{##1}}}
\expandafter\def\csname PY@tok@gs\endcsname{\let\PY@bf=\textbf}
\expandafter\def\csname PY@tok@gr\endcsname{\def\PY@tc##1{\textcolor[rgb]{1.00,0.00,0.00}{##1}}}
\expandafter\def\csname PY@tok@cm\endcsname{\let\PY@it=\textit\def\PY@tc##1{\textcolor[rgb]{0.25,0.50,0.50}{##1}}}
\expandafter\def\csname PY@tok@vg\endcsname{\def\PY@tc##1{\textcolor[rgb]{0.10,0.09,0.49}{##1}}}
\expandafter\def\csname PY@tok@m\endcsname{\def\PY@tc##1{\textcolor[rgb]{0.40,0.40,0.40}{##1}}}
\expandafter\def\csname PY@tok@mh\endcsname{\def\PY@tc##1{\textcolor[rgb]{0.40,0.40,0.40}{##1}}}
\expandafter\def\csname PY@tok@go\endcsname{\def\PY@tc##1{\textcolor[rgb]{0.53,0.53,0.53}{##1}}}
\expandafter\def\csname PY@tok@ge\endcsname{\let\PY@it=\textit}
\expandafter\def\csname PY@tok@vc\endcsname{\def\PY@tc##1{\textcolor[rgb]{0.10,0.09,0.49}{##1}}}
\expandafter\def\csname PY@tok@il\endcsname{\def\PY@tc##1{\textcolor[rgb]{0.40,0.40,0.40}{##1}}}
\expandafter\def\csname PY@tok@cs\endcsname{\let\PY@it=\textit\def\PY@tc##1{\textcolor[rgb]{0.25,0.50,0.50}{##1}}}
\expandafter\def\csname PY@tok@cp\endcsname{\def\PY@tc##1{\textcolor[rgb]{0.74,0.48,0.00}{##1}}}
\expandafter\def\csname PY@tok@gi\endcsname{\def\PY@tc##1{\textcolor[rgb]{0.00,0.63,0.00}{##1}}}
\expandafter\def\csname PY@tok@gh\endcsname{\let\PY@bf=\textbf\def\PY@tc##1{\textcolor[rgb]{0.00,0.00,0.50}{##1}}}
\expandafter\def\csname PY@tok@ni\endcsname{\let\PY@bf=\textbf\def\PY@tc##1{\textcolor[rgb]{0.60,0.60,0.60}{##1}}}
\expandafter\def\csname PY@tok@nl\endcsname{\def\PY@tc##1{\textcolor[rgb]{0.63,0.63,0.00}{##1}}}
\expandafter\def\csname PY@tok@nn\endcsname{\let\PY@bf=\textbf\def\PY@tc##1{\textcolor[rgb]{0.00,0.00,1.00}{##1}}}
\expandafter\def\csname PY@tok@no\endcsname{\def\PY@tc##1{\textcolor[rgb]{0.53,0.00,0.00}{##1}}}
\expandafter\def\csname PY@tok@na\endcsname{\def\PY@tc##1{\textcolor[rgb]{0.49,0.56,0.16}{##1}}}
\expandafter\def\csname PY@tok@nb\endcsname{\def\PY@tc##1{\textcolor[rgb]{0.00,0.50,0.00}{##1}}}
\expandafter\def\csname PY@tok@nc\endcsname{\let\PY@bf=\textbf\def\PY@tc##1{\textcolor[rgb]{0.00,0.00,1.00}{##1}}}
\expandafter\def\csname PY@tok@nd\endcsname{\def\PY@tc##1{\textcolor[rgb]{0.67,0.13,1.00}{##1}}}
\expandafter\def\csname PY@tok@ne\endcsname{\let\PY@bf=\textbf\def\PY@tc##1{\textcolor[rgb]{0.82,0.25,0.23}{##1}}}
\expandafter\def\csname PY@tok@nf\endcsname{\def\PY@tc##1{\textcolor[rgb]{0.00,0.00,1.00}{##1}}}
\expandafter\def\csname PY@tok@si\endcsname{\let\PY@bf=\textbf\def\PY@tc##1{\textcolor[rgb]{0.73,0.40,0.53}{##1}}}
\expandafter\def\csname PY@tok@s2\endcsname{\def\PY@tc##1{\textcolor[rgb]{0.73,0.13,0.13}{##1}}}
\expandafter\def\csname PY@tok@vi\endcsname{\def\PY@tc##1{\textcolor[rgb]{0.10,0.09,0.49}{##1}}}
\expandafter\def\csname PY@tok@nt\endcsname{\let\PY@bf=\textbf\def\PY@tc##1{\textcolor[rgb]{0.00,0.50,0.00}{##1}}}
\expandafter\def\csname PY@tok@nv\endcsname{\def\PY@tc##1{\textcolor[rgb]{0.10,0.09,0.49}{##1}}}
\expandafter\def\csname PY@tok@s1\endcsname{\def\PY@tc##1{\textcolor[rgb]{0.73,0.13,0.13}{##1}}}
\expandafter\def\csname PY@tok@sh\endcsname{\def\PY@tc##1{\textcolor[rgb]{0.73,0.13,0.13}{##1}}}
\expandafter\def\csname PY@tok@sc\endcsname{\def\PY@tc##1{\textcolor[rgb]{0.73,0.13,0.13}{##1}}}
\expandafter\def\csname PY@tok@sx\endcsname{\def\PY@tc##1{\textcolor[rgb]{0.00,0.50,0.00}{##1}}}
\expandafter\def\csname PY@tok@bp\endcsname{\def\PY@tc##1{\textcolor[rgb]{0.00,0.50,0.00}{##1}}}
\expandafter\def\csname PY@tok@c1\endcsname{\let\PY@it=\textit\def\PY@tc##1{\textcolor[rgb]{0.25,0.50,0.50}{##1}}}
\expandafter\def\csname PY@tok@kc\endcsname{\let\PY@bf=\textbf\def\PY@tc##1{\textcolor[rgb]{0.00,0.50,0.00}{##1}}}
\expandafter\def\csname PY@tok@c\endcsname{\let\PY@it=\textit\def\PY@tc##1{\textcolor[rgb]{0.25,0.50,0.50}{##1}}}
\expandafter\def\csname PY@tok@mf\endcsname{\def\PY@tc##1{\textcolor[rgb]{0.40,0.40,0.40}{##1}}}
\expandafter\def\csname PY@tok@err\endcsname{\def\PY@bc##1{\setlength{\fboxsep}{0pt}\fcolorbox[rgb]{1.00,0.00,0.00}{1,1,1}{\strut ##1}}}
\expandafter\def\csname PY@tok@kd\endcsname{\let\PY@bf=\textbf\def\PY@tc##1{\textcolor[rgb]{0.00,0.50,0.00}{##1}}}
\expandafter\def\csname PY@tok@ss\endcsname{\def\PY@tc##1{\textcolor[rgb]{0.10,0.09,0.49}{##1}}}
\expandafter\def\csname PY@tok@sr\endcsname{\def\PY@tc##1{\textcolor[rgb]{0.73,0.40,0.53}{##1}}}
\expandafter\def\csname PY@tok@mo\endcsname{\def\PY@tc##1{\textcolor[rgb]{0.40,0.40,0.40}{##1}}}
\expandafter\def\csname PY@tok@kn\endcsname{\let\PY@bf=\textbf\def\PY@tc##1{\textcolor[rgb]{0.00,0.50,0.00}{##1}}}
\expandafter\def\csname PY@tok@mi\endcsname{\def\PY@tc##1{\textcolor[rgb]{0.40,0.40,0.40}{##1}}}
\expandafter\def\csname PY@tok@gp\endcsname{\let\PY@bf=\textbf\def\PY@tc##1{\textcolor[rgb]{0.00,0.00,0.50}{##1}}}
\expandafter\def\csname PY@tok@o\endcsname{\def\PY@tc##1{\textcolor[rgb]{0.40,0.40,0.40}{##1}}}
\expandafter\def\csname PY@tok@kr\endcsname{\let\PY@bf=\textbf\def\PY@tc##1{\textcolor[rgb]{0.00,0.50,0.00}{##1}}}
\expandafter\def\csname PY@tok@s\endcsname{\def\PY@tc##1{\textcolor[rgb]{0.73,0.13,0.13}{##1}}}
\expandafter\def\csname PY@tok@kp\endcsname{\def\PY@tc##1{\textcolor[rgb]{0.00,0.50,0.00}{##1}}}
\expandafter\def\csname PY@tok@w\endcsname{\def\PY@tc##1{\textcolor[rgb]{0.73,0.73,0.73}{##1}}}
\expandafter\def\csname PY@tok@kt\endcsname{\def\PY@tc##1{\textcolor[rgb]{0.69,0.00,0.25}{##1}}}
\expandafter\def\csname PY@tok@ow\endcsname{\let\PY@bf=\textbf\def\PY@tc##1{\textcolor[rgb]{0.67,0.13,1.00}{##1}}}
\expandafter\def\csname PY@tok@sb\endcsname{\def\PY@tc##1{\textcolor[rgb]{0.73,0.13,0.13}{##1}}}
\expandafter\def\csname PY@tok@k\endcsname{\let\PY@bf=\textbf\def\PY@tc##1{\textcolor[rgb]{0.00,0.50,0.00}{##1}}}
\expandafter\def\csname PY@tok@se\endcsname{\let\PY@bf=\textbf\def\PY@tc##1{\textcolor[rgb]{0.73,0.40,0.13}{##1}}}
\expandafter\def\csname PY@tok@sd\endcsname{\let\PY@it=\textit\def\PY@tc##1{\textcolor[rgb]{0.73,0.13,0.13}{##1}}}

\def\PYZbs{\char`\\}
\def\PYZus{\char`\_}
\def\PYZob{\char`\{}
\def\PYZcb{\char`\}}
\def\PYZca{\char`\^}
\def\PYZam{\char`\&}
\def\PYZlt{\char`\<}
\def\PYZgt{\char`\>}
\def\PYZsh{\char`\#}
\def\PYZpc{\char`\%}
\def\PYZdl{\char`\$}
\def\PYZhy{\char`\-}
\def\PYZsq{\char`\'}
\def\PYZdq{\char`\"}
\def\PYZti{\char`\~}
% for compatibility with earlier versions
\def\PYZat{@}
\def\PYZlb{[}
\def\PYZrb{]}
\makeatother


    % Exact colors from NB
    \definecolor{incolor}{rgb}{0.0, 0.0, 0.5}
    \definecolor{outcolor}{rgb}{0.545, 0.0, 0.0}



    
    % Prevent overflowing lines due to hard-to-break entities
    \sloppy 
    % Setup hyperref package
    \hypersetup{
      breaklinks=true,  % so long urls are correctly broken across lines
      colorlinks=true,
      urlcolor=blue,
      linkcolor=darkorange,
      citecolor=darkgreen,
      }
    % Slightly bigger margins than the latex defaults
    
    \geometry{verbose,tmargin=1in,bmargin=1in,lmargin=1in,rmargin=1in}
    
    

    \begin{document}
    
    
    \maketitle
    
    

    
    \begin{Verbatim}[commandchars=\\\{\}]
{\color{incolor}In [{\color{incolor}187}]:} \PY{o}{\PYZpc{}}\PY{k}{pylab} \PY{n}{inline}
          \PY{k+kn}{from} \PY{n+nn}{sympy} \PY{k+kn}{import} \PY{o}{*}
          \PY{k+kn}{from} \PY{n+nn}{IPython.display} \PY{k+kn}{import} \PY{n}{Image}
          \PY{n}{init\PYZus{}printing}\PY{p}{(}\PY{n}{use\PYZus{}latex}\PY{o}{=}\PY{l+s}{\PYZdq{}}\PY{l+s}{mathjax}\PY{l+s}{\PYZdq{}}\PY{p}{)}
\end{Verbatim}

    \begin{Verbatim}[commandchars=\\\{\}]
Populating the interactive namespace from numpy and matplotlib
    \end{Verbatim}

    \begin{Verbatim}[commandchars=\\\{\}]
WARNING: pylab import has clobbered these variables: ['prod', 'Circle', 'power', 'diag', 'sinh', 'trunc', 'binomial', 'plot', 'eye', 'det', 'tan', 'product', 'roots', 'vectorize', 'sin', 'plotting', 'zeros', 'cosh', 'conjugate', 'take', 'solve', 'trace', 'beta', 'ones', 'transpose', 'cos', 'interactive', 'diff', 'invert', 'tanh', 'Polygon', 'reshape', 'sqrt', 'floor', 'source', 'add', 'multinomial', 'test', 'poly', 'mod', 'sign', 'gamma', 'log', 'var', 'seterr', 'flatten', 'nan', 'pi', 'exp']
`\%matplotlib` prevents importing * from pylab and numpy
    \end{Verbatim}


    \section{Rotazione del subreflettore intorno al fuoco}


    \begin{Verbatim}[commandchars=\\\{\}]
{\color{incolor}In [{\color{incolor}188}]:} \PY{n}{Image}\PY{p}{(}\PY{l+s}{\PYZdq{}}\PY{l+s}{files/rot\PYZus{}sub.png}\PY{l+s}{\PYZdq{}}\PY{p}{)}
\end{Verbatim}
\texttt{\color{outcolor}Out[{\color{outcolor}188}]:}
    
    \begin{center}
    \adjustimage{max size={0.9\linewidth}{0.9\paperheight}}{RotazioniSubreflettore_files/RotazioniSubreflettore_2_0.png}
    \end{center}
    { \hspace*{\fill} \\}
    

    \begin{Verbatim}[commandchars=\\\{\}]
{\color{incolor}In [{\color{incolor}188}]:} 
\end{Verbatim}

    Sia dato il sistema descritto il figura in cui: * \textbf{O} rappresenta
il centro del subriflettore in posizione 0 * \textbf{F} rappresenta il
fulcro del parboloide * \textbf{D} è la distanza tra il subriflettore e
il fuoco nella posizione 0 * \$ \theta = \widehat{OFP}\$ , angolo di
tilt del subriflettore necessario ad inquadrare un feed in vertex *
\(|FO| = |FP| = \textbf{D}\)

Vogliamo trovare le compensazioni \textbf{Z} e \textbf{X} da applicare
al movimento del subriflettore per mantenerlo centrato sull'asse di
fuoco senza introdurre errori di puntamento.

    Essendo il triangolo \(OFP\) isoscele abbiamo che:
\[ \alpha = \frac{\pi - \theta}{2} = \frac{\pi}{2} - \frac{\theta}{2} \]

    da cui possiamo ricavare l'ampiezza dell'angolo \(\beta\):
\[ \beta = \frac{\pi}{2} - \alpha = \frac{\pi}{2} - \frac{\pi}{2} + \frac{\theta}{2} = \frac{\theta}{2} \]

    Possiamo quindi calcolare \textbf{X} e \textbf{Z} sapendo che:
\[ |PO| = 2 * (D * \sin{\frac{\theta}{2}}) \] E sfruttando il fatto che
\(PBO\) è rettangolo in \(B\):
\[ X = |PO| * \cos{\beta} = 2 * D * \sin{\frac{\theta}{2}} * \cos{\frac{\theta}{2}} = D* \sin{\theta} \]
\[ Z = |PO| * \sin{\beta} = 2 * D * \sin{\frac{\theta}{2}} * \sin{\frac{\theta}{2}} = 2 * D * (\sin{\frac{\theta}{2}})^2\]

    \begin{Verbatim}[commandchars=\\\{\}]
{\color{incolor}In [{\color{incolor}189}]:} \PY{n}{d}\PY{p}{,} \PY{n}{t}\PY{p}{,} \PY{n}{z}\PY{p}{,} \PY{n}{x} \PY{o}{=} \PY{n}{symbols}\PY{p}{(}\PY{l+s}{\PYZdq{}}\PY{l+s}{D theta z x}\PY{l+s}{\PYZdq{}}\PY{p}{)}
          \PY{n}{z} \PY{o}{=} \PY{l+m+mi}{2} \PY{o}{*} \PY{n}{d} \PY{o}{*} \PY{p}{(}\PY{n}{sin}\PY{p}{(}\PY{n}{t}\PY{o}{/}\PY{l+m+mi}{2}\PY{p}{)}\PY{o}{*}\PY{o}{*}\PY{l+m+mi}{2}\PY{p}{)}
          \PY{n}{x} \PY{o}{=} \PY{n}{d} \PY{o}{*} \PY{n}{sin}\PY{p}{(}\PY{n}{t}\PY{p}{)}
\end{Verbatim}

    Sapendo che \$ D = 310.7mm \$ calcolato geometricamente per il
subreflettore possiamo calcolare il variare di \textbf{X} e \textbf{Z}
al variare dell'angolo in vertex

    \begin{Verbatim}[commandchars=\\\{\}]
{\color{incolor}In [{\color{incolor}190}]:} \PY{n}{xaxis} \PY{o}{=} \PY{n}{linspace}\PY{p}{(}\PY{l+m+mi}{0}\PY{p}{,} \PY{n}{radians}\PY{p}{(}\PY{l+m+mi}{8}\PY{p}{)}\PY{p}{,} \PY{l+m+mi}{1000}\PY{p}{)}
          \PY{n}{xaxis\PYZus{}labels} \PY{o}{=} \PY{n}{linspace}\PY{p}{(}\PY{l+m+mi}{0}\PY{p}{,} \PY{l+m+mi}{8}\PY{p}{,} \PY{l+m+mi}{1000}\PY{p}{)}
          \PY{n}{z\PYZus{}at\PYZus{}fixed\PYZus{}d} \PY{o}{=} \PY{l+m+mi}{2} \PY{o}{*} \PY{l+m+mf}{310.7} \PY{o}{*} \PY{n}{np}\PY{o}{.}\PY{n}{sin}\PY{p}{(}\PY{n}{xaxis}\PY{o}{/}\PY{l+m+mi}{2}\PY{p}{)}\PY{o}{*}\PY{o}{*}\PY{l+m+mi}{2}
          \PY{n}{x\PYZus{}at\PYZus{}fixed\PYZus{}d} \PY{o}{=} \PY{l+m+mf}{310.7} \PY{o}{*} \PY{n}{np}\PY{o}{.}\PY{n}{sin}\PY{p}{(}\PY{n}{xaxis}\PY{p}{)}
          \PY{c}{\PYZsh{}z\PYZus{}at\PYZus{}fixed\PYZus{}d = array([N(z.subs(D, 310.7).subs(t, \PYZus{}theta)) for \PYZus{}theta in xaxis])}
          \PY{c}{\PYZsh{}x\PYZus{}at\PYZus{}fixed\PYZus{}d = array([N(x.subs(D, 310.7).subs(t, \PYZus{}theta)) for \PYZus{}theta in xaxis])}
\end{Verbatim}

    Dai plot seguenti vediamo come nella posizione 0 del sureflettore una
rotazione di un angolo \(\theta\) attorno al fuoco rappresenti una
piccola correzione in Z e una più consistente correzione in X

    \begin{Verbatim}[commandchars=\\\{\}]
{\color{incolor}In [{\color{incolor}191}]:} \PY{n}{pylab}\PY{o}{.}\PY{n}{title}\PY{p}{(}\PY{l+s}{\PYZdq{}}\PY{l+s}{Z correction at D = 310.7}\PY{l+s}{\PYZdq{}}\PY{p}{)}
          \PY{n}{pylab}\PY{o}{.}\PY{n}{xlabel}\PY{p}{(}\PY{l+s}{\PYZdq{}}\PY{l+s}{Theta deg.}\PY{l+s}{\PYZdq{}}\PY{p}{)}
          \PY{n}{pylab}\PY{o}{.}\PY{n}{ylabel}\PY{p}{(}\PY{l+s}{\PYZdq{}}\PY{l+s}{Z mm}\PY{l+s}{\PYZdq{}}\PY{p}{)}
          \PY{n}{pylab}\PY{o}{.}\PY{n}{grid}\PY{p}{(}\PY{n+nb+bp}{True}\PY{p}{)}
          \PY{n}{pylab}\PY{o}{.}\PY{n}{plot}\PY{p}{(}\PY{n}{xaxis\PYZus{}labels}\PY{p}{,} \PY{n}{z\PYZus{}at\PYZus{}fixed\PYZus{}d}\PY{p}{,} \PY{l+s}{\PYZsq{}}\PY{l+s}{b\PYZhy{}}\PY{l+s}{\PYZsq{}}\PY{p}{,} 
                     \PY{n}{xaxis\PYZus{}labels}\PY{p}{[}\PY{l+m+mi}{500}\PY{p}{]}\PY{p}{,} \PY{n}{z\PYZus{}at\PYZus{}fixed\PYZus{}d}\PY{p}{[}\PY{l+m+mi}{500}\PY{p}{]}\PY{p}{,} \PY{l+s}{\PYZsq{}}\PY{l+s}{bo}\PY{l+s}{\PYZsq{}}\PY{p}{,} 
                     \PY{n}{markersize} \PY{o}{=} \PY{l+m+mi}{12}\PY{p}{)}
\end{Verbatim}

            \begin{Verbatim}[commandchars=\\\{\}]
{\color{outcolor}Out[{\color{outcolor}191}]:} [<matplotlib.lines.Line2D at 0xee374d0>,
           <matplotlib.lines.Line2D at 0xee37750>]
\end{Verbatim}
        
    \begin{center}
    \adjustimage{max size={0.9\linewidth}{0.9\paperheight}}{RotazioniSubreflettore_files/RotazioniSubreflettore_12_1.png}
    \end{center}
    { \hspace*{\fill} \\}
    
    \begin{Verbatim}[commandchars=\\\{\}]
{\color{incolor}In [{\color{incolor}192}]:} \PY{n}{pylab}\PY{o}{.}\PY{n}{title}\PY{p}{(}\PY{l+s}{\PYZdq{}}\PY{l+s}{X correction at D = 310.7}\PY{l+s}{\PYZdq{}}\PY{p}{)}
          \PY{n}{pylab}\PY{o}{.}\PY{n}{xlabel}\PY{p}{(}\PY{l+s}{\PYZdq{}}\PY{l+s}{Theta deg.}\PY{l+s}{\PYZdq{}}\PY{p}{)}
          \PY{n}{pylab}\PY{o}{.}\PY{n}{ylabel}\PY{p}{(}\PY{l+s}{\PYZdq{}}\PY{l+s}{X mm}\PY{l+s}{\PYZdq{}}\PY{p}{)}
          \PY{n}{pylab}\PY{o}{.}\PY{n}{grid}\PY{p}{(}\PY{n+nb+bp}{True}\PY{p}{)}
          \PY{n}{pylab}\PY{o}{.}\PY{n}{plot}\PY{p}{(}\PY{n}{xaxis\PYZus{}labels}\PY{p}{,} \PY{n}{x\PYZus{}at\PYZus{}fixed\PYZus{}d}\PY{p}{,} \PY{l+s}{\PYZsq{}}\PY{l+s}{g\PYZhy{}}\PY{l+s}{\PYZsq{}}\PY{p}{,} 
                     \PY{n}{xaxis\PYZus{}labels}\PY{p}{[}\PY{l+m+mi}{500}\PY{p}{]}\PY{p}{,} \PY{n}{x\PYZus{}at\PYZus{}fixed\PYZus{}d}\PY{p}{[}\PY{l+m+mi}{500}\PY{p}{]}\PY{p}{,} \PY{l+s}{\PYZsq{}}\PY{l+s}{go}\PY{l+s}{\PYZsq{}}\PY{p}{,} 
                     \PY{n}{markersize} \PY{o}{=} \PY{l+m+mi}{12}\PY{p}{)}
\end{Verbatim}

            \begin{Verbatim}[commandchars=\\\{\}]
{\color{outcolor}Out[{\color{outcolor}192}]:} [<matplotlib.lines.Line2D at 0xf074ad0>,
           <matplotlib.lines.Line2D at 0xf074d50>]
\end{Verbatim}
        
    \begin{center}
    \adjustimage{max size={0.9\linewidth}{0.9\paperheight}}{RotazioniSubreflettore_files/RotazioniSubreflettore_13_1.png}
    \end{center}
    { \hspace*{\fill} \\}
    
    Presupponiamo ora di inquadrare un ricevitore con un angolo di
\(\theta = 4^{\circ}\) e di voler effettuare un focheggiamento muovendo
il subreflettore verso il ricevitore di una quantità che facciamo
variare tra 0 e 18cm, calcolati come \$ 3\lambda  @ 5 GHz\$ . Anche in
questo caso possiamo controllare il variare delle correzioni applicate
in \textbf{Z} e \textbf{X} al variare della distanza del subreflettore
dal fuoco.

    \begin{Verbatim}[commandchars=\\\{\}]
{\color{incolor}In [{\color{incolor}193}]:} \PY{n}{xaxis} \PY{o}{=} \PY{n}{linspace}\PY{p}{(}\PY{l+m+mi}{0}\PY{p}{,} \PY{l+m+mi}{180}\PY{p}{,} \PY{l+m+mi}{1000}\PY{p}{)}
          \PY{n}{z\PYZus{}at\PYZus{}fixed\PYZus{}t} \PY{o}{=} \PY{n}{array}\PY{p}{(}\PY{p}{[}\PY{n}{N}\PY{p}{(}\PY{n}{z}\PY{o}{.}\PY{n}{subs}\PY{p}{(}\PY{n}{t}\PY{p}{,} \PY{n}{radians}\PY{p}{(}\PY{l+m+mi}{4}\PY{p}{)}\PY{p}{)}\PY{o}{.}\PY{n}{subs}\PY{p}{(}\PY{n}{D}\PY{p}{,} \PY{n}{\PYZus{}d}\PY{p}{)}\PY{p}{)} \PY{k}{for} \PY{n}{\PYZus{}d} \PY{o+ow}{in} \PY{n}{xaxis}\PY{p}{]}\PY{p}{)}
          \PY{n}{x\PYZus{}at\PYZus{}fixed\PYZus{}t} \PY{o}{=} \PY{n}{array}\PY{p}{(}\PY{p}{[}\PY{n}{N}\PY{p}{(}\PY{n}{x}\PY{o}{.}\PY{n}{subs}\PY{p}{(}\PY{n}{t}\PY{p}{,} \PY{n}{radians}\PY{p}{(}\PY{l+m+mi}{4}\PY{p}{)}\PY{p}{)}\PY{o}{.}\PY{n}{subs}\PY{p}{(}\PY{n}{D}\PY{p}{,} \PY{n}{\PYZus{}d}\PY{p}{)}\PY{p}{)} \PY{k}{for} \PY{n}{\PYZus{}d} \PY{o+ow}{in} \PY{n}{xaxis}\PY{p}{]}\PY{p}{)}
\end{Verbatim}

    \begin{Verbatim}[commandchars=\\\{\}]
{\color{incolor}In [{\color{incolor}194}]:} \PY{n}{pylab}\PY{o}{.}\PY{n}{title}\PY{p}{(}\PY{l+s}{\PYZdq{}}\PY{l+s}{Z correction at theta = 4 deg.}\PY{l+s}{\PYZdq{}}\PY{p}{)}
          \PY{n}{pylab}\PY{o}{.}\PY{n}{xlabel}\PY{p}{(}\PY{l+s}{\PYZdq{}}\PY{l+s}{D mm}\PY{l+s}{\PYZdq{}}\PY{p}{)}
          \PY{n}{pylab}\PY{o}{.}\PY{n}{ylabel}\PY{p}{(}\PY{l+s}{\PYZdq{}}\PY{l+s}{Z mm}\PY{l+s}{\PYZdq{}}\PY{p}{)}
          \PY{n}{pylab}\PY{o}{.}\PY{n}{grid}\PY{p}{(}\PY{n+nb+bp}{True}\PY{p}{)}
          \PY{n}{pylab}\PY{o}{.}\PY{n}{plot}\PY{p}{(}\PY{n}{xaxis}\PY{p}{,} \PY{n}{z\PYZus{}at\PYZus{}fixed\PYZus{}t}\PY{p}{,} \PY{l+s}{\PYZsq{}}\PY{l+s}{b\PYZhy{}}\PY{l+s}{\PYZsq{}}\PY{p}{)}
\end{Verbatim}

            \begin{Verbatim}[commandchars=\\\{\}]
{\color{outcolor}Out[{\color{outcolor}194}]:} [<matplotlib.lines.Line2D at 0xf2de290>]
\end{Verbatim}
        
    \begin{center}
    \adjustimage{max size={0.9\linewidth}{0.9\paperheight}}{RotazioniSubreflettore_files/RotazioniSubreflettore_16_1.png}
    \end{center}
    { \hspace*{\fill} \\}
    
    \begin{Verbatim}[commandchars=\\\{\}]
{\color{incolor}In [{\color{incolor}195}]:} \PY{n}{pylab}\PY{o}{.}\PY{n}{title}\PY{p}{(}\PY{l+s}{\PYZdq{}}\PY{l+s}{X correction at theta = 4 deg.}\PY{l+s}{\PYZdq{}}\PY{p}{)}
          \PY{n}{pylab}\PY{o}{.}\PY{n}{xlabel}\PY{p}{(}\PY{l+s}{\PYZdq{}}\PY{l+s}{D mm}\PY{l+s}{\PYZdq{}}\PY{p}{)}
          \PY{n}{pylab}\PY{o}{.}\PY{n}{ylabel}\PY{p}{(}\PY{l+s}{\PYZdq{}}\PY{l+s}{X mm}\PY{l+s}{\PYZdq{}}\PY{p}{)}
          \PY{n}{pylab}\PY{o}{.}\PY{n}{grid}\PY{p}{(}\PY{n+nb+bp}{True}\PY{p}{)}
          \PY{n}{pylab}\PY{o}{.}\PY{n}{plot}\PY{p}{(}\PY{n}{xaxis}\PY{p}{,} \PY{n}{x\PYZus{}at\PYZus{}fixed\PYZus{}t}\PY{p}{,} \PY{l+s}{\PYZsq{}}\PY{l+s}{g\PYZhy{}}\PY{l+s}{\PYZsq{}}\PY{p}{)}
\end{Verbatim}

            \begin{Verbatim}[commandchars=\\\{\}]
{\color{outcolor}Out[{\color{outcolor}195}]:} [<matplotlib.lines.Line2D at 0xf364cd0>]
\end{Verbatim}
        
    \begin{center}
    \adjustimage{max size={0.9\linewidth}{0.9\paperheight}}{RotazioniSubreflettore_files/RotazioniSubreflettore_17_1.png}
    \end{center}
    { \hspace*{\fill} \\}
    

    \section{Tilt del subreflettore}


    \begin{Verbatim}[commandchars=\\\{\}]
{\color{incolor}In [{\color{incolor}196}]:} \PY{n}{Image}\PY{p}{(}\PY{l+s}{\PYZdq{}}\PY{l+s}{files/tilt\PYZus{}sub2.png}\PY{l+s}{\PYZdq{}}\PY{p}{)}
\end{Verbatim}
\texttt{\color{outcolor}Out[{\color{outcolor}196}]:}
    
    \begin{center}
    \adjustimage{max size={0.9\linewidth}{0.9\paperheight}}{RotazioniSubreflettore_files/RotazioniSubreflettore_19_0.png}
    \end{center}
    { \hspace*{\fill} \\}
    

    Sia dato il sistema in figura in cui: * \(AB\) è un lato del triangolo
subreflettore (vista B in Fig. 3.1.2 pag. 90 ``Matematica di sistema'' 4
di 4) * \(|AB| = 2068mm\) * \(\varphi\) è l'angolo di cui vgliamo
ruotare attorno al suo asse il subreflettore * \(BC\) rappresenta il
posizionamento del subreflettore una volta compiuta la rotazione

    \textbf{NOTA}: \emph{Per semplicità consideriamo il caso di rotazione
sull'asse X applicando un angolo \(\theta_y\) che si riflette sul solo
movimento di due attuatori. In questo caso sto solo cercando di
quantificare un possibile errore, in realtà B si dovrebbe muovere verso
l'alto mentre A verso il basso, in questo caso direi che quello che ci
interessa è la differenza tra i due. }

    Ragionando sui triangoli e sfruttando il fatto che \(|AB| = |BC|\)
possiamo desumere che: * \(\omega = \Omega\) perchè alterni-interni
\[ \Omega = \omega = \frac{\pi - \varphi}{2} = \frac{\pi}{2} - \frac{\varphi}{2} \]
\[ \sin{\Omega} = \sin{\omega} = \cos{\frac{\varphi}{2}} \]
\[ \cos{\Omega} = \cos{\omega} = \sin{\frac{\varphi}{2}} \]

    Ragionando a questo punto sui lati otteniamo che:

    \[ |AC| = 2*|AB|*\sin(\frac{\varphi}{2}) \]
\[ |AE| = |AC| * \sin{\omega} = 2*|AB|*\sin{\frac{\varphi}{2}} * \cos{\frac{\varphi}{2}} = 2*|AB|*\frac{\sin{\varphi}}{2} = |AB|*\sin{\varphi}\]
\[ |EC| = |AC| * \cos{\omega} = 2*|AB|*\sin{\frac{\varphi}{2}} * \sin{\frac{\varphi}{2}}  = 2*|AB|*(\sin{\frac{\varphi}{2}})^2 = |AB|*(1 - \cos{\varphi}) \]
\[ |ED| = |EC| * \tan{\varphi} = |AB|* \tan{\varphi} - |AB|*\cos{\varphi}*\frac{\sin{\varphi}}{\cos{\varphi}} = |AB| * (\tan{\varphi} - \sin{\varphi})\]
\[ |AD| = |AB| * \tan{\varphi} \]

    Que e ci interessa è che quindi quando noi vogliamo comandare un tilt di
\(\varphi\) e diamo come posizione comandata \(|AD|\) in realtà dovremmo
dare un valore di \(|AC|\) , per cui:
\[ |AD| - |AC| = |AB| * \tan{\varphi} - 2*|AB|*\sin{\frac{\varphi}{2}} = |AB|*(\tan{\varphi} - 2*\sin{\frac{\varphi}{2}}) \]

    Fissanso quindi come da ipotesi \(|AB| = 2098mm\) calcoliamo l'errore al
variare di \(\varphi\):

    \begin{Verbatim}[commandchars=\\\{\}]
{\color{incolor}In [{\color{incolor}199}]:} \PY{n}{xaxis} \PY{o}{=} \PY{n}{linspace}\PY{p}{(}\PY{l+m+mi}{0}\PY{p}{,} \PY{n}{radians}\PY{p}{(}\PY{l+m+mi}{8}\PY{p}{)}\PY{p}{,} \PY{l+m+mi}{1000}\PY{p}{)}
          \PY{n}{xaxis\PYZus{}labels} \PY{o}{=} \PY{n}{linspace}\PY{p}{(}\PY{l+m+mi}{0}\PY{p}{,} \PY{l+m+mi}{8}\PY{p}{,} \PY{l+m+mi}{1000}\PY{p}{)}
          \PY{n}{error} \PY{o}{=} \PY{p}{(}\PY{l+m+mi}{2098}\PY{o}{/}\PY{l+m+mi}{2}\PY{p}{)} \PY{o}{*} \PY{p}{(}\PY{n}{np}\PY{o}{.}\PY{n}{tan}\PY{p}{(}\PY{n}{xaxis}\PY{p}{)} \PY{o}{\PYZhy{}} \PY{l+m+mi}{2}\PY{o}{*}\PY{n}{np}\PY{o}{.}\PY{n}{sin}\PY{p}{(}\PY{n}{xaxis}\PY{o}{/}\PY{l+m+mi}{2}\PY{p}{)}\PY{p}{)}
\end{Verbatim}

    \begin{Verbatim}[commandchars=\\\{\}]
{\color{incolor}In [{\color{incolor}201}]:} \PY{n}{pylab}\PY{o}{.}\PY{n}{title}\PY{p}{(}\PY{l+s}{\PYZdq{}}\PY{l+s}{Correzione attuatore vs. angolo di tilt}\PY{l+s}{\PYZdq{}}\PY{p}{)}
          \PY{n}{pylab}\PY{o}{.}\PY{n}{xlabel}\PY{p}{(}\PY{l+s}{\PYZdq{}}\PY{l+s}{Phi deg.}\PY{l+s}{\PYZdq{}}\PY{p}{)}
          \PY{n}{pylab}\PY{o}{.}\PY{n}{ylabel}\PY{p}{(}\PY{l+s}{\PYZdq{}}\PY{l+s}{Error mm}\PY{l+s}{\PYZdq{}}\PY{p}{)}
          \PY{n}{pylab}\PY{o}{.}\PY{n}{grid}\PY{p}{(}\PY{n+nb+bp}{True}\PY{p}{)}
          \PY{n}{pylab}\PY{o}{.}\PY{n}{plot}\PY{p}{(}\PY{n}{xaxis\PYZus{}labels}\PY{p}{,} \PY{n}{error}\PY{p}{,} \PY{l+s}{\PYZsq{}}\PY{l+s}{b\PYZhy{}}\PY{l+s}{\PYZsq{}}\PY{p}{,} 
                     \PY{n}{xaxis\PYZus{}labels}\PY{p}{[}\PY{l+m+mi}{500}\PY{p}{]}\PY{p}{,} \PY{n}{error}\PY{p}{[}\PY{l+m+mi}{500}\PY{p}{]}\PY{p}{,} \PY{l+s}{\PYZsq{}}\PY{l+s}{bo}\PY{l+s}{\PYZsq{}}\PY{p}{,} 
                     \PY{n}{markersize} \PY{o}{=} \PY{l+m+mi}{12}\PY{p}{)}
\end{Verbatim}

            \begin{Verbatim}[commandchars=\\\{\}]
{\color{outcolor}Out[{\color{outcolor}201}]:} [<matplotlib.lines.Line2D at 0xfa16310>,
           <matplotlib.lines.Line2D at 0xfa16590>]
\end{Verbatim}
        
    \begin{center}
    \adjustimage{max size={0.9\linewidth}{0.9\paperheight}}{RotazioniSubreflettore_files/RotazioniSubreflettore_28_1.png}
    \end{center}
    { \hspace*{\fill} \\}
    
    \begin{Verbatim}[commandchars=\\\{\}]
{\color{incolor}In [{\color{incolor}}]:} 
\end{Verbatim}

    \begin{Verbatim}[commandchars=\\\{\}]
{\color{incolor}In [{\color{incolor}}]:} 
\end{Verbatim}


    % Add a bibliography block to the postdoc
    
    
    
    \end{document}
